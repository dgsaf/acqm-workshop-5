\documentclass{article}

% - style template
\usepackage{base}
\geometry{a4paper, margin = 1in}

% - title, author, etc.
\title{PHYS4000 - Workshop 5}
\author{Tom Ross - 1834 2884}
\date{\today}

% - headers
\pagestyle{fancy}
\fancyhf{}
\rhead{\theauthor}
\chead{}
\lhead{\thetitle}
\rfoot{\thepage}
\cfoot{}
\lfoot{}

% - document
\begin{document}

\tableofcontents

\listoffigures

\listoftables

\clearpage

\section{e-H V-Matrix}
\label{sec:e-h-v-matrix}

\subsection{Implementation}
\label{sec:implementation}

We denote the hydrogen target states by $\ket{\phi_{i}}$, and we denote the
electron projectile states by $\ket{\vb{k}}$ corresponding to continuum waves
with energy $\tfrac{1}{2} k^{2}$, where $k = \lrnorm{\vb{k}}$.
We shall work in the s-wave model; that is, we only consider target states
with $\ell_{i} = 0$ and $m_{i} = 0$, and continuum states with
\begin{equation*}
  \braket{\vb{r}}{\vb{k}}
  =
  N_{k}
  \sin\lr{r k}
  .
\end{equation*}
Note that we shall neglect the normalisation constants $N_{k}$ henceforth.
We calculate potential matrix elements of the form
\begin{alignat*}{2}
  V_{f, i}^{\lr{S}}\lr{k', k}
  {}={}
  &
  \mel{k', \phi_{f}}{\hat{V}^{S}}{\phi_{i}, k}
  \\
  {}={}
  &
  \mel
  {k', \phi_{f}}
  {\hat{V} - \lr{-1}^{S}\lr{E- \hat{H}}\hat{P}_{r}}
  {\phi_{i}, k}
  \\
  {}={}
  &
  \mel{k', \phi_{f}}{\hat{V}_{1} + \hat{V}_{1, 2}}{\phi_{i}, k}
  -
  \lr{-1}^{S}
  \mel{k', \phi_{f}}{E - \hat{H}}{k, \phi_{i}}
  \\
  {}={}
  &
  D_{f, i}\lr{k', k}
  -
  \lr{-1}^{S}
  X_{f, i}\lr{k', k}
\end{alignat*}
where $D_{f, i}\lr{k', k}$ is the direct matrix element, $X_{f, i}\lr{k', k}$ is
the exchange matrix element, and where $\hat{V}_{1}$ is the electron-nuclear
potential of the form
\begin{equation*}
  \hat{V}_{1}
  =
  -
  \dfrac{1}{r_{1}}
  ,
\end{equation*}
and where $\hat{V}_{1, 2}$ is the electron-electron potential of the form
\begin{equation*}
  \hat{V}_{1, 2}
  =
  \dfrac{1}{\lrnorm{\vb{r}_{1} - \vb{r}_{2}}}
  =
  \sum_{\lambda = 0}^{\infty}
  \dfrac{4\pi}{2\lambda + 1}
  \dfrac{r_{<}^{\lambda}}{r_{>}^{\lambda + 1}}
  \sum_{\mu = -\lambda}^{\lambda}
  Y_{\lambda}^{\mu}\lr{\Omega_{1}}
  {Y_{\lambda}^{\mu}}^{*}\lr{\Omega_{2}}
\end{equation*}
where $r_{<} = \min\lr{r_{1}, r_{2}}$, $r_{>} = \max\lr{r_{1}, r_{2}}$,
and where $Y_{\lambda}^{\mu}$ are the spherical harmonics.
However, within the s-wave model this potential reduces to the form
\begin{equation*}
  \hat{V}_{1, 2}
  =
  \dfrac{1}{r_{>}}
  =
  \dfrac{1}{\max\lr{r_{1}, r_{2}}}
  .
\end{equation*}

\subsubsection{Calculation of Direct Matrix Elements $D_{f, i}\lr{k', k}$}
\label{sec:calc-dir-me}

The direct matrix elements are of the form
\begin{equation*}
  D_{f, i}\lr{k', k}
  =
  \mel{k', \phi_{f}}{\hat{V}_{1} + \hat{V}_{1, 2}}{\phi_{i}, k}
\end{equation*}
where
\begin{alignat*}{2}
  \mel{k', \phi_{f}}{\hat{V}_{1}}{\phi_{i}, k}
  {}={}
  &
  \mel{k'}{\hat{V}_{1}}{k}
  \braket{\phi_{f}}{\phi_{i}}
  \\
  {}={}
  &
  -
  \int_{0}^{\infty}
  {
    \dfrac{1}{r_{1}}
    \sin\lr{k' r_{1}}
    \sin\lr{k r_{1}}
  }
  \dd{r_{1}}
  \int_{0}^{\infty}
  {
    \phi_{f}\lr{r_{2}}
    \phi_{i}\lr{r_{2}}
  }
  \dd{r_{2}}
  \\
  {}={}
  &
  -
  \int_{0}^{\infty}
  {
    \dfrac{1}{r_{1}}
    \sin\lr{k' r_{1}}
    \sin\lr{k r_{1}}
  }
  \dd{r_{1}}
  \delta_{f, i}
\end{alignat*}
and where
\begin{alignat*}{2}
  \mel{k', \phi_{f}}{\hat{V}_{1, 2}}{\phi_{i}, k}
  {}={}
  &
  \int_{0}^{\infty}
  \int_{0}^{\infty}
  {
    \sin\lr{k' r_{1}}
    \phi_{f}\lr{r_{2}}
    \dfrac{1}{\max\lr{r_{1}, r_{2}}}
    \sin\lr{k r_{1}}
    \phi_{i}\lr{r_{2}}
  }
  \dd{r_{1}}
  \dd{r_{2}}
  \\
  {}={}
  &
  \int_{0}^{\infty}
  {
    \sin\lr{k' r_{1}}
    \sin\lr{k r_{1}}
    \lr[\bigg]
    {
      \int_{0}^{\infty}
      {
        \dfrac{1}{\max\lr{r_{1}, r_{2}}}
        \phi_{f}\lr{r_{2}}
        \phi_{i}\lr{r_{2}}
      }
      \dd{r_{2}}
    }
  }
  \dd{r_{1}}
  \\
  {}={}
  &
  \int_{0}^{\infty}
  {
    \sin\lr{k' r_{1}}
    \sin\lr{k r_{1}}
    \lr[\bigg]
    {
      \dfrac{1}{r_{1}}
      \int_{0}^{r_{1}}
      {
        \phi_{f}\lr{r_{2}}
        \phi_{i}\lr{r_{2}}
      }
      \dd{r_{2}}
      +
      \int_{r_{1}}^{\infty}
      {
        \dfrac{1}{r_{2}}
        \phi_{f}\lr{r_{2}}
        \phi_{i}\lr{r_{2}}
      }
      \dd{r_{2}}
    }
  }
  \dd{r_{1}}
  .
\end{alignat*}

\subsubsection{Calculation of Exchange Matrix Elements $X_{f, i}\lr{k', k}$}
\label{sec:calc-exc-me}

The exchange matrix elements are of the form
\begin{alignat*}{2}
  X_{f, i}\lr{k', k}
  {}={}
  &
  \mel{k', \phi_{f}}{E - \hat{H}}{k, \phi_{i}}
  \\
  {}={}
  &
  \lr
  {
    E
    -
    \tfrac{1}{2} k'^{2}
    -
    \tfrac{1}{2} k^{2}
  }
  \braket{k'}{\phi_{i}}
  \braket{\phi_{f}}{k}
  \\
  &
  {}-{}
  \mel{k'}{\hat{V}_{1}}{\phi_{i}}
  \braket{\phi_{f}}{k}
  -
  \braket{k'}{\phi_{i}}
  \mel{\phi_{f}}{\hat{V}_{2}}{k}
  \\
  &
  {}-{}
  \mel{k', \phi_{f}}{\hat{V}_{1, 2}}{k \phi_{i}}
\end{alignat*}
where the one-electron inner products are of the form
\begin{equation*}
  \braket{f}{g}
  =
  \int_{0}^{\infty}
  {
    f\lr{r}
    g\lr{r}
  }
  \dd{r}
\end{equation*}
and where the two-electron inner product is of the form
\begin{alignat*}{2}
  \mel{k', \phi_{f}}{\hat{V}_{1, 2}}{k \phi_{i}}
  {}={}
  &
  \int_{0}^{\infty}
  \int_{0}^{\infty}
  {
    \sin\lr{k' r_{1}}
    \phi_{f}\lr{r_{2}}
    \dfrac{1}{\max\lr{r_{1}, r_{2}}}
    \phi_{i}\lr{r_{1}}
    \sin\lr{k r_{2}}
  }
  \dd{r_{1}}
  \dd{r_{2}}
  \\
  {}={}
  &
  \int_{0}^{\infty}
  {
    \sin\lr{k' r_{1}}
    \phi_{i}\lr{r_{1}}
    \lr[\bigg]
    {
      \int_{0}^{\infty}
      {
        \dfrac{1}{\max\lr{r_{1}, r_{2}}}
        \phi_{f}\lr{r_{2}}
        \sin\lr{k r_{2}}
      }
      \dd{r_{2}}
    }
  }
  \dd{r_{1}}
  \\
  {}={}
  &
  \int_{0}^{\infty}
  {
    \sin\lr{k' r_{1}}
    \phi_{i}\lr{r_{1}}
    \lr[\bigg]
    {
      \dfrac{1}{r_{1}}
      \int_{0}^{r_{1}}
      {
        \phi_{f}\lr{r_{2}}
        \sin\lr{k r_{2}}
      }
      \dd{r_{2}}
      +
      \int_{r_{1}}^{\infty}
      {
        \dfrac{1}{r_{2}}
        \phi_{f}\lr{r_{2}}
        \sin\lr{k r_{2}}
      }
      \dd{r_{2}}
    }
  }
  \dd{r_{1}}
  \\
\end{alignat*}

\subsubsection{Evaluation of Integrals}
\label{sec:eval-int}

We suppose that the radial functions are to be plotted on a radial grid of the
form $\mathcal{R}_{n_{r}} = \lrset{k \delta_{r}}_{k = 1}^{n_{r}}$ for
$n_{r} > 0$ and small $\delta_{r} > 0$, and with a corresponding set of weights
$\mathcal{W}_{n_{r}} = \lrset{w_{k}}_{k = 1}^{n_{r}}$ such that
\begin{equation*}
  \int_{0}^{\infty}
  {
    f\lr{r}
  }
  \dd{r}
  =
  \lim_{n_{r} \to \infty}
  \sum_{k = 1}^{n_{r}}
  {
    w_{k}
    f\lr{r_{k}}
  }
  =
  \lim_{n_{r} \to \infty}
  \sum_{k = 1}^{n_{r}}
  {
    w_{k}
    f_{k}
  }
  \approx
  \sum_{k = 1}^{n_{r}}
  {
    w_{k}
    f_{k}
  }
  .
\end{equation*}
The one-electron integrals, of the form $\braket{f}{g}$, are then
evaluated in the following manner
\begin{equation*}
  \braket{f}{g}
  =
  \int_{0}^{\infty}
  {
    f\lr{r}
    g\lr{r}
  }
  \dd{r}
  \approx
  \sum_{k = 1}^{n_{r}}
  {
    w_{k}
    f_{k}
    g_{k}
  }
  .
\end{equation*}
The two electron integrals $\bra{F} {} \ket{G}$ where
\begin{equation*}
  \bra{F} {} \ket{G}
  =
  \int_{0}^{\infty}
  {
    F\lr{r_{1}}
    \lr[\bigg]
    {
      \dfrac{1}{r_{1}}
      \int_{0}^{r_{1}}
      {
        G\lr{r_{2}}
      }
      \dd{r_{2}}
      +
      \int_{r_{1}}^{\infty}
      {
        \dfrac{1}{r_{2}}
        G\lr{r_{2}}
      }
      \dd{r_{2}}
    }
  }
  \dd{r_{1}}
  ,
\end{equation*}
are evaluated in the following manner
\begin{equation*}
  \bra{F} {} \ket{G}
  \approx
  \sum_{k = 1}^{n_{r}}
  {
    w_{k}
    F_{k}
    \lr[\bigg]
    {
      \dfrac{1}{r_{k}}
      A_{k}
      +
      B_{k}
    }
  }
\end{equation*}
where
\begin{equation*}
  A_{k}
  =
  \sum_{m = 1}^{k}
  {
    w_{m}
    G_{m}
  }
  =
  \begin{cases}
    A_{k - 1}
    +
    w_{k}
    G_{k}
    &
    \qq{for}
    k = 2, \dotsc, n_{r}
    \\
    w_{k}
    G_{k}
    &
    \qq{for}
    k = 1
  \end{cases}
\end{equation*}
and
\begin{equation*}
  B_{k}
  =
  \sum_{m = k}^{n_{r}}
  {
    \dfrac{1}{r_{m}}
    w_{m}
    G_{m}
  }
  =
  \begin{cases}
    B_{k + 1}
    +
    \tfrac{1}{r_{k}}
    w_{k}
    G_{k}
    &
    \qq{for}
    k = 1, \dotsc, n_{r} - 1
    \\
    \tfrac{1}{r_{k}}
    w_{k}
    G_{k}
    &
    \qq{for}
    k = n_{r}
  \end{cases}
  .
\end{equation*}

\subsection{Direct Matrix Elements}
\label{sec:dir-me}

\begin{figure}[h]
  \begin{center}
    \input{figure_1s_1s_dir.tex}
  \end{center}
  \caption[Direct Matrix Elements 1s-1s]{
    ...
  }
  \label{fig:dir-me-1s-1s}
\end{figure}

\begin{figure}[h]
  \begin{center}
    \input{figure_1s_2s_dir.tex}
  \end{center}
  \caption[Direct Matrix Elements 1s-2s]{
    ...
  }
  \label{fig:dir-me-1s-2s}
\end{figure}

\begin{figure}[h]
  \begin{center}
    \input{figure_1s_3s_dir.tex}
  \end{center}
  \caption[Direct Matrix Elements 1s-3s]{
    ...
  }
  \label{fig:dir-me-1s-3s}
\end{figure}

\clearpage

\subsection{Exchange Matrix Elements}
\label{sec:exc-me}

\begin{figure}[h]
  \begin{center}
    \input{figure_1s_1s_exc.tex}
  \end{center}
  \caption[Exchange Matrix Elements 1s-1s]{
    ...
  }
  \label{fig:exc-me-1s-1s}
\end{figure}

\begin{figure}[h]
  \begin{center}
    \input{figure_1s_2s_exc.tex}
  \end{center}
  \caption[Exchange Matrix Elements 1s-2s]{
    ...
  }
  \label{fig:exc-me-1s-2s}
\end{figure}

\begin{figure}[h]
  \begin{center}
    \input{figure_1s_3s_exc.tex}
  \end{center}
  \caption[Exchange Matrix Elements 1s-3s]{
    ...
  }
  \label{fig:exc-me-1s-3s}
\end{figure}

\clearpage

\subsection{On-Shell Matrix Elements}
\label{sec:on-me}

\begin{figure}[h]
  \begin{center}
    \input{figure_1s_1s_on.tex}
  \end{center}
  \caption[On-Shell Matrix Elements 1s-1s]{
    ...
  }
  \label{fig:on-me-1s-1s}
\end{figure}

\begin{figure}[h]
  \begin{center}
    \input{figure_1s_2s_on.tex}
  \end{center}
  \caption[On-Shell Matrix Elements 1s-2s]{
    ...
  }
  \label{fig:on-me-1s-2s}
\end{figure}

\begin{figure}[h]
  \begin{center}
    \input{figure_1s_3s_on.tex}
  \end{center}
  \caption[On-Shell Matrix Elements 1s-3s]{
    ...
  }
  \label{fig:on-me-1s-3s}
\end{figure}

\clearpage

\section{$V_{1 2}$ Potential in S-Wave Model}
\label{sec:v12-swave}

\section{Reduced CCC Code}
\label{sec:reduced-ccc-code}

\end{document}
