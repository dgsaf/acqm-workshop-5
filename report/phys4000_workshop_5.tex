\documentclass{article}

% - style template
\usepackage{base}
\geometry{a4paper, margin = 1in}

% - title, author, etc.
\title{PHYS4000 - Workshop 5}
\author{Tom Ross - 1834 2884}
\date{\today}

% - headers
\pagestyle{fancy}
\fancyhf{}
\rhead{\theauthor}
\chead{}
\lhead{\thetitle}
\rfoot{\thepage}
\cfoot{}
\lfoot{}

% - document
\begin{document}

\tableofcontents

\listoffigures

\listoftables

\clearpage

The entire code repository used to calculate the data for this report can be
found at \url{https://github.com/dgsaf/acqm-workshop-5}.
Note that \autoref{sec:reduced-ccc-code} has not yet been attempted - it may be
at a later date if time permits.

\section{e-H V-Matrix}
\label{sec:e-h-v-matrix}

\subsection{Implementation}
\label{sec:implementation}

We denote the hydrogen target states by $\ket{\phi_{i}}$, and we denote the
electron projectile states by $\ket{\vb{k}}$ corresponding to continuum waves
with energy $\tfrac{1}{2} k^{2}$, where $k = \lrnorm{\vb{k}}$.
We shall work in the s-wave model; that is, we only consider target states
with $\ell_{i} = 0$ and $m_{i} = 0$, and continuum states with
\begin{equation*}
  \braket{\vb{r}}{\vb{k}}
  =
  N_{k}
  \sin\lr{r k}
  .
\end{equation*}
Note that we shall neglect the normalisation constants $N_{k}$ henceforth.
We calculate potential matrix elements of the form
\begin{alignat*}{2}
  V_{f, i}^{\lr{S}}\lr{k', k}
  {}={}
  &
  \mel{k', \phi_{f}}{\hat{V}^{S}}{\phi_{i}, k}
  \\
  {}={}
  &
  \mel
  {k', \phi_{f}}
  {\hat{V} - \lr{-1}^{S}\lr{E- \hat{H}}\hat{P}_{r}}
  {\phi_{i}, k}
  \\
  {}={}
  &
  \mel{k', \phi_{f}}{\hat{V}_{1} + \hat{V}_{1, 2}}{\phi_{i}, k}
  -
  \lr{-1}^{S}
  \mel{k', \phi_{f}}{E - \hat{H}}{k, \phi_{i}}
  \\
  {}={}
  &
  D_{f, i}\lr{k', k}
  -
  \lr{-1}^{S}
  X_{f, i}\lr{k', k}
\end{alignat*}
where $D_{f, i}\lr{k', k}$ is the direct matrix element, $X_{f, i}\lr{k', k}$ is
the exchange matrix element, and where $\hat{V}_{1}$ is the electron-nuclear
potential of the form
\begin{equation*}
  \hat{V}_{1}
  =
  -
  \dfrac{1}{r_{1}}
  ,
\end{equation*}
and where $\hat{V}_{1, 2}$ is the electron-electron potential of the form
\begin{equation*}
  \hat{V}_{1, 2}
  =
  \dfrac{1}{\lrnorm{\vb{r}_{1} - \vb{r}_{2}}}
  =
  \sum_{\lambda = 0}^{\infty}
  \dfrac{4\pi}{2\lambda + 1}
  \dfrac{r_{<}^{\lambda}}{r_{>}^{\lambda + 1}}
  \sum_{\mu = -\lambda}^{\lambda}
  Y_{\lambda}^{\mu}\lr{\Omega_{1}}
  {Y_{\lambda}^{\mu}}^{*}\lr{\Omega_{2}}
\end{equation*}
where $r_{<} = \min\lr{r_{1}, r_{2}}$, $r_{>} = \max\lr{r_{1}, r_{2}}$,
and where $Y_{\lambda}^{\mu}$ are the spherical harmonics.
However, within the s-wave model this potential reduces to the form
\begin{equation*}
  \hat{V}_{1, 2}
  =
  \dfrac{1}{r_{>}}
  =
  \dfrac{1}{\max\lr{r_{1}, r_{2}}}
  .
\end{equation*}

\subsubsection{Calculation of Direct Matrix Elements $D_{f, i}\lr{k', k}$}
\label{sec:calc-dir-me}

The direct matrix elements are of the form
\begin{equation*}
  D_{f, i}\lr{k', k}
  =
  \mel{k', \phi_{f}}{\hat{V}_{1} + \hat{V}_{1, 2}}{\phi_{i}, k}
\end{equation*}
where
\begin{alignat*}{2}
  \mel{k', \phi_{f}}{\hat{V}_{1}}{\phi_{i}, k}
  {}={}
  &
  \mel{k'}{\hat{V}_{1}}{k}
  \braket{\phi_{f}}{\phi_{i}}
  \\
  {}={}
  &
  -
  \int_{0}^{\infty}
  {
    \dfrac{1}{r_{1}}
    \sin\lr{k' r_{1}}
    \sin\lr{k r_{1}}
  }
  \dd{r_{1}}
  \int_{0}^{\infty}
  {
    \phi_{f}\lr{r_{2}}
    \phi_{i}\lr{r_{2}}
  }
  \dd{r_{2}}
  \\
  {}={}
  &
  -
  \int_{0}^{\infty}
  {
    \dfrac{1}{r_{1}}
    \sin\lr{k' r_{1}}
    \sin\lr{k r_{1}}
  }
  \dd{r_{1}}
  \delta_{f, i}
\end{alignat*}
and where
\begin{alignat*}{2}
  \mel{k', \phi_{f}}{\hat{V}_{1, 2}}{\phi_{i}, k}
  {}={}
  &
  \int_{0}^{\infty}
  \int_{0}^{\infty}
  {
    \sin\lr{k' r_{1}}
    \phi_{f}\lr{r_{2}}
    \dfrac{1}{\max\lr{r_{1}, r_{2}}}
    \sin\lr{k r_{1}}
    \phi_{i}\lr{r_{2}}
  }
  \dd{r_{1}}
  \dd{r_{2}}
  \\
  {}={}
  &
  \int_{0}^{\infty}
  {
    \sin\lr{k' r_{1}}
    \sin\lr{k r_{1}}
    \lr[\bigg]
    {
      \int_{0}^{\infty}
      {
        \dfrac{1}{\max\lr{r_{1}, r_{2}}}
        \phi_{f}\lr{r_{2}}
        \phi_{i}\lr{r_{2}}
      }
      \dd{r_{2}}
    }
  }
  \dd{r_{1}}
  \\
  {}={}
  &
  \int_{0}^{\infty}
  {
    \sin\lr{k' r_{1}}
    \sin\lr{k r_{1}}
    \lr[\bigg]
    {
      \dfrac{1}{r_{1}}
      \int_{0}^{r_{1}}
      {
        \phi_{f}\lr{r_{2}}
        \phi_{i}\lr{r_{2}}
      }
      \dd{r_{2}}
      +
      \int_{r_{1}}^{\infty}
      {
        \dfrac{1}{r_{2}}
        \phi_{f}\lr{r_{2}}
        \phi_{i}\lr{r_{2}}
      }
      \dd{r_{2}}
    }
  }
  \dd{r_{1}}
  .
\end{alignat*}

\subsubsection{Calculation of Exchange Matrix Elements $X_{f, i}\lr{k', k}$}
\label{sec:calc-exc-me}

The exchange matrix elements are of the form
\begin{alignat*}{2}
  X_{f, i}\lr{k', k}
  {}={}
  &
  \mel{k', \phi_{f}}{E - \hat{H}}{k, \phi_{i}}
  \\
  {}={}
  &
  \lr
  {
    E
    -
    \tfrac{1}{2} k'^{2}
    -
    \tfrac{1}{2} k^{2}
  }
  \braket{k'}{\phi_{i}}
  \braket{\phi_{f}}{k}
  \\
  &
  {}-{}
  \mel{k'}{\hat{V}_{1}}{\phi_{i}}
  \braket{\phi_{f}}{k}
  -
  \braket{k'}{\phi_{i}}
  \mel{\phi_{f}}{\hat{V}_{2}}{k}
  \\
  &
  {}-{}
  \mel{k', \phi_{f}}{\hat{V}_{1, 2}}{k \phi_{i}}
\end{alignat*}
where the one-electron inner products are of the form
\begin{equation*}
  \braket{f}{g}
  =
  \int_{0}^{\infty}
  {
    f\lr{r}
    g\lr{r}
  }
  \dd{r}
\end{equation*}
and where the two-electron inner product is of the form
\begin{alignat*}{2}
  \mel{k', \phi_{f}}{\hat{V}_{1, 2}}{k \phi_{i}}
  {}={}
  &
  \int_{0}^{\infty}
  \int_{0}^{\infty}
  {
    \sin\lr{k' r_{1}}
    \phi_{f}\lr{r_{2}}
    \dfrac{1}{\max\lr{r_{1}, r_{2}}}
    \phi_{i}\lr{r_{1}}
    \sin\lr{k r_{2}}
  }
  \dd{r_{1}}
  \dd{r_{2}}
  \\
  {}={}
  &
  \int_{0}^{\infty}
  {
    \sin\lr{k' r_{1}}
    \phi_{i}\lr{r_{1}}
    \lr[\bigg]
    {
      \int_{0}^{\infty}
      {
        \dfrac{1}{\max\lr{r_{1}, r_{2}}}
        \phi_{f}\lr{r_{2}}
        \sin\lr{k r_{2}}
      }
      \dd{r_{2}}
    }
  }
  \dd{r_{1}}
  \\
  {}={}
  &
  \int_{0}^{\infty}
  {
    \sin\lr{k' r_{1}}
    \phi_{i}\lr{r_{1}}
    \lr[\bigg]
    {
      \dfrac{1}{r_{1}}
      \int_{0}^{r_{1}}
      {
        \phi_{f}\lr{r_{2}}
        \sin\lr{k r_{2}}
      }
      \dd{r_{2}}
      +
      \int_{r_{1}}^{\infty}
      {
        \dfrac{1}{r_{2}}
        \phi_{f}\lr{r_{2}}
        \sin\lr{k r_{2}}
      }
      \dd{r_{2}}
    }
  }
  \dd{r_{1}}
  \\
\end{alignat*}

\subsubsection{Evaluation of Integrals}
\label{sec:eval-int}

We suppose that the radial functions are to be plotted on a radial grid of the
form $\mathcal{R} = \lrset{k \delta_{r}}_{k = 1}^{n_{r}}$ for
$n_{r} > 0$ and small $\delta_{r} > 0$, and with a corresponding set of weights
$\mathcal{W} = \lrset{w_{k}}_{k = 1}^{n_{r}}$ such that
\begin{equation*}
  \int_{0}^{\infty}
  {
    f\lr{r}
  }
  \dd{r}
  =
  \lim_{n_{r} \to \infty}
  \sum_{k = 1}^{n_{r}}
  {
    w_{k}
    f\lr{r_{k}}
  }
  =
  \lim_{n_{r} \to \infty}
  \sum_{k = 1}^{n_{r}}
  {
    w_{k}
    f_{k}
  }
  \approx
  \sum_{k = 1}^{n_{r}}
  {
    w_{k}
    f_{k}
  }
  .
\end{equation*}
The one-electron integrals, of the form $\braket{f}{g}$, are then
evaluated in the following manner
\begin{equation*}
  \braket{f}{g}
  =
  \int_{0}^{\infty}
  {
    f\lr{r}
    g\lr{r}
  }
  \dd{r}
  \approx
  \sum_{k = 1}^{n_{r}}
  {
    w_{k}
    f_{k}
    g_{k}
  }
  .
\end{equation*}
The two electron integrals $\bra{F} {} \ket{G}$ where
\begin{equation*}
  \bra{F} {} \ket{G}
  =
  \int_{0}^{\infty}
  {
    F\lr{r_{1}}
    \lr[\bigg]
    {
      \dfrac{1}{r_{1}}
      \int_{0}^{r_{1}}
      {
        G\lr{r_{2}}
      }
      \dd{r_{2}}
      +
      \int_{r_{1}}^{\infty}
      {
        \dfrac{1}{r_{2}}
        G\lr{r_{2}}
      }
      \dd{r_{2}}
    }
  }
  \dd{r_{1}}
  ,
\end{equation*}
are evaluated in the following manner
\begin{equation*}
  \bra{F} {} \ket{G}
  \approx
  \sum_{k = 1}^{n_{r}}
  {
    w_{k}
    F_{k}
    \lr[\bigg]
    {
      \dfrac{1}{r_{k}}
      A_{k}
      +
      B_{k}
    }
  }
\end{equation*}
where
\begin{equation*}
  A_{k}
  =
  \sum_{m = 1}^{k}
  {
    w_{m}
    G_{m}
  }
  =
  \begin{cases}
    A_{k - 1}
    +
    w_{k}
    G_{k}
    &
    \qq{for}
    k = 2, \dotsc, n_{r}
    \\
    w_{k}
    G_{k}
    &
    \qq{for}
    k = 1
  \end{cases}
\end{equation*}
and
\begin{equation*}
  B_{k}
  =
  \sum_{m = k}^{n_{r}}
  {
    \dfrac{1}{r_{m}}
    w_{m}
    G_{m}
  }
  =
  \begin{cases}
    B_{k + 1}
    +
    \tfrac{1}{r_{k}}
    w_{k}
    G_{k}
    &
    \qq{for}
    k = 1, \dotsc, n_{r} - 1
    \\
    \tfrac{1}{r_{k}}
    w_{k}
    G_{k}
    &
    \qq{for}
    k = n_{r}
  \end{cases}
  .
\end{equation*}

\clearpage

For these calculations, we have used: a radial grid of the form
$\mathcal{R} = \lrset{i \delta_{r}}_{i = 1}^{n_{r}}$, with
$\delta_{r} = 0.01$ and $\max\lr{\mathcal{R}} = 100$,
and a momentum grid of the form
$\mathcal{K} = \lrset{i \delta_{k}}_{i = 1}^{n_{k}}$, with
$\delta_{k} = 0.025$ and $\max\lr{\mathcal{K}} = 5$.

\subsection{Direct Matrix Elements}
\label{sec:dir-me}

\begin{figure}[h]
  \begin{center}
    \input{figure_10_1s_1s_dir.tex}
  \end{center}
  \caption[Direct Matrix Elements 1s-1s]{
    The $D_{f, i}\lr{k', k}$ direct matrix elements (shown in red) are presented
    for the $1s \to 1s$ transition, with $E = \SI{10}{\eV}$.
  }
  \label{fig:dir-me-1s-1s}
\end{figure}

\begin{figure}[h]
  \begin{center}
    \input{figure_10_1s_2s_dir.tex}
  \end{center}
  \caption[Direct Matrix Elements 1s-2s]{
    The $D_{f, i}\lr{k', k}$ direct matrix elements (shown in red) are presented
    for the $1s \to 2s$ transition, with $E = \SI{10}{\eV}$.
  }
  \label{fig:dir-me-1s-2s}
\end{figure}

\begin{figure}[h]
  \begin{center}
    \input{figure_10_1s_3s_dir.tex}
  \end{center}
  \caption[Direct Matrix Elements 1s-3s]{
    The $D_{f, i}\lr{k', k}$ direct matrix elements (shown in red) are presented
    for the $1s \to 3s$ transition, with $E = \SI{10}{\eV}$.
  }
  \label{fig:dir-me-1s-3s}
\end{figure}

\clearpage

\subsection{Exchange Matrix Elements}
\label{sec:exc-me}

\subsubsection{$X_{f, i}\lr{k', k}$ for $E = \SI{10}{\eV}$}
\label{sec:exc-me-10}

\begin{figure}[h]
  \begin{center}
    \input{figure_10_1s_1s_exc.tex}
  \end{center}
  \caption[Exchange Matrix Elements 1s-1s]{
    The $X_{f, i}\lr{k', k}$ exchange matrix elements (shown in blue) are
    presented for the $1s \to 1s$ transition, with $E = \SI{10}{\eV}$.
  }
  \label{fig:exc-me-1s-1s}
\end{figure}

\begin{figure}[h]
  \begin{center}
    \input{figure_10_1s_2s_exc.tex}
  \end{center}
  \caption[Exchange Matrix Elements 1s-2s]{
    The $X_{f, i}\lr{k', k}$ exchange matrix elements (shown in blue) are
    presented for the $1s \to 2s$ transition, with $E = \SI{10}{\eV}$.
  }
  \label{fig:exc-me-1s-2s}
\end{figure}

\begin{figure}[h]
  \begin{center}
    \input{figure_10_1s_3s_exc.tex}
  \end{center}
  \caption[Exchange Matrix Elements 1s-3s]{
    The $X_{f, i}\lr{k', k}$ exchange matrix elements (shown in blue) are
    presented for the $1s \to 3s$ transition, with $E = \SI{10}{\eV}$.
  }
  \label{fig:exc-me-1s-3s}
\end{figure}

\clearpage

\subsubsection{$X_{f, i}\lr{k', k}$ for $E = \SI{1}{\eV}$}
\label{sec:exc-me-1}

\begin{figure}[h]
  \begin{center}
    \input{figure_1_1s_1s_exc.tex}
  \end{center}
  \caption[Exchange Matrix Elements 1s-1s]{
    The $X_{f, i}\lr{k', k}$ exchange matrix elements (shown in blue) are
    presented for the $1s \to 1s$ transition, with $E = \SI{1}{\eV}$.
  }
  \label{fig:exc-me-1s-1s}
\end{figure}

\begin{figure}[h]
  \begin{center}
    \input{figure_1_1s_2s_exc.tex}
  \end{center}
  \caption[Exchange Matrix Elements 1s-2s]{
    The $X_{f, i}\lr{k', k}$ exchange matrix elements (shown in blue) are
    presented for the $1s \to 2s$ transition, with $E = \SI{1}{\eV}$.
  }
  \label{fig:exc-me-1s-2s}
\end{figure}

\begin{figure}[h]
  \begin{center}
    \input{figure_1_1s_3s_exc.tex}
  \end{center}
  \caption[Exchange Matrix Elements 1s-3s]{
    The $X_{f, i}\lr{k', k}$ exchange matrix elements (shown in blue) are
    presented for the $1s \to 3s$ transition, with $E = \SI{1}{\eV}$.
  }
  \label{fig:exc-me-1s-3s}
\end{figure}

\clearpage

\subsection{On-Shell Matrix Elements}
\label{sec:on-me}

For the transisiton $[i \to f]$, the on-shell energy is of the form
\begin{equation*}
  \epsilon_{i}
  +
  \dfrac{1}{2}
  k^{2}
  =
  E
  =
  \epsilon_{f}
  +
  \dfrac{1}{2}
  k'^{2}
\end{equation*}
whence the on-shell value for $k'$ is defined by
\begin{equation*}
  \dfrac{1}{2}
  k'^{2}
  =
  E
  -
  \epsilon_{f}
  =
  \dfrac{1}{2}
  k^{2}
  +
  \epsilon_{i}
  -
  \epsilon_{f}
\end{equation*}
where for the case of a hydrogen target, with
$\epsilon_{n} = -\tfrac{1}{2n^{2}}$, and $i = 1$, we have have that
\begin{equation*}
  k'
  =
  \sqrt
  {
    k^{2}
    -
    \dfrac{f^{2} - 1}{f^{2}}
  }
  .
\end{equation*}
Note that transitions are forbidden for
\begin{equation*}
  k^{2}
  <
  \dfrac{f^{2} - 1}{f^{2}}
  =
  1
  -
  \dfrac{1}{f^{2}}
  .
\end{equation*}

\begin{figure}[h]
  \begin{center}
    \input{figure_10_1s_1s_on.tex}
  \end{center}
  \caption[On-Shell Matrix Elements 1s-1s]{
    The $D_{f, i}\lr{k', k}$ direct and $X_{f, i}\lr{k', k}$ exchange on-shell
    matrix elements (shown in red and blue respectively) are presented for the
    $1s \to 1s$ transition, and are compared with their respective analytic
    expressions (shown in black).
  }
  \label{fig:on-me-1s-1s}
\end{figure}

\begin{figure}[h]
  \begin{center}
    \input{figure_10_1s_2s_on.tex}
  \end{center}
  \caption[On-Shell Matrix Elements 1s-2s]{
    The $D_{f, i}\lr{k', k}$ direct and $X_{f, i}\lr{k', k}$ exchange on-shell
    matrix elements (shown in red and blue respectively) are presented for the
    $1s \to 2s$ transition, and are compared with their respective analytic
    expressions (shown in black).
    Note that the numerically calculated and analytic expressions differ by a
    phase of $\lr{-1}$ (which remains a puzzle to me).
  }
  \label{fig:on-me-1s-2s}
\end{figure}

\begin{figure}[h]
  \begin{center}
    \input{figure_10_1s_3s_on.tex}
  \end{center}
  \caption[On-Shell Matrix Elements 1s-3s]{
    The $D_{f, i}\lr{k', k}$ direct and $X_{f, i}\lr{k', k}$ exchange on-shell
    matrix elements (shown in red and blue respectively) are presented for the
    $1s \to 3s$ transition, and are compared with their respective analytic
    expressions (shown in black).
    Note that the numerically calculated and analytic expressions differ by a
    phase of $\lr{-1}$ (which remains a puzzle to me).
  }
  \label{fig:on-me-1s-3s}
\end{figure}

\clearpage

\section{$V_{1 2}$ Potential in S-Wave Model}
\label{sec:v12-swave}

We note that the general form for a continuum wave
$\ket{\vb{k}} = \ket{k, \ell, m}$ is
\begin{equation*}
  \braket{\vb{r}}{\vb{k}}
  =
  \dfrac{1}{r}
  u_{\ell}\lr{r; k}
  Y_{\ell}^{m}\lr{\Omega}
  ,
\end{equation*}
and that the general form for the hydrogen target states
$\ket{\phi_{i}} = \ket{\phi_{n_{i}, \ell_{i}, m_{i}}}$ is
\begin{equation*}
  \braket{\vb{r}}{\phi_{i}}
  =
  \dfrac{1}{r}
  \phi_{i}\lr{r}
  Y_{\ell_{i}}^{m_{i}}\lr{\Omega}
  .
\end{equation*}
Recall that the electron-electron potential is of the form
\begin{equation*}
  \hat{V}_{1, 2}
  =
  \dfrac{1}{\lrnorm{\vb{r}_{1} - \vb{r}_{2}}}
  =
  \sum_{\lambda = 0}^{\infty}
  \dfrac{4\pi}{2\lambda + 1}
  \dfrac{r_{<}^{\lambda}}{r_{>}^{\lambda + 1}}
  \sum_{\mu = -\lambda}^{\lambda}
  Y_{\lambda}^{\mu}\lr{\Omega_{1}}
  {Y_{\lambda}^{\mu}}^{*}\lr{\Omega_{2}}
\end{equation*}
where $r_{<} = \min\lr{r_{1}, r_{2}}$, $r_{>} = \max\lr{r_{1}, r_{2}}$,
and where $Y_{\lambda}^{\mu}$ are the spherical harmonics.
We now consider the form of the two-electron term in
$D_{f, i}\lr{\vb{k}', \vb{k}}$,
\begin{equation*}
  \mel{\vb{k}', \phi_{f}}{\hat{V}_{1, 2}}{\phi_{i}, \vb{k}}
  =
  \int
  \int
  {
    \braket{\vb{r}_{1}}{\vb{k}'}^{*}
    \braket{\vb{r}_{2}}{\phi_{f}}^{*}
    \dfrac{1}{\lrnorm{\vb{r}_{1} - \vb{r}_{2}}}
    \braket{\vb{r}_{1}}{\vb{k}}
    \braket{\vb{r}_{2}}{\phi_{i}}
  }
  \dd{\vb{r}_{1}}
  \dd{\vb{r}_{2}}
\end{equation*}
for which the partial wave expansion is of the form
\begin{alignat*}{3}
  &
  \mel{\vb{k}', \phi_{f}}{\hat{V}_{1, 2}}{\phi_{i}, \vb{k}}
  &{}={}&
  \sum_{\lambda = 0}^{\infty}
  \dfrac{4\pi}{2\lambda + 1}
  \sum_{\mu = -\lambda}^{\lambda}
  &&
  \int_{0}^{\infty}
  \int_{0}^{\infty}
  {
    \dfrac{u_{\ell'}\lr{r_{1}; k'}}{r_{1}}
    \dfrac{\phi_{f}\lr{r_{2}}}{r_{2}}
    \dfrac{r_{<}^{\lambda}}{r_{>}^{\lambda + 1}}
    \dfrac{\phi_{i}\lr{r_{2}}}{r_{2}}
    \dfrac{u_{\ell}\lr{r_{1}; k}}{r_{1}}
  }
  r_{1}^{2}
  r_{2}^{2}
  \dd{r_{1}}
  \dd{r_{2}}
  \\
  &
  {}
  &&
  {}
  &&
  {}\times{}
  \int_{S^{2}}
  {
    {Y_{\ell'}^{m'}}^{*}\lr{\Omega_{1}}
    Y_{\lambda}^{\mu}\lr{\Omega_{1}}
    Y_{\ell}^{m}\lr{\Omega_{1}}
  }
  \dd{\Omega_{1}}
  \\
  &
  {}
  &&
  {}
  &&
  {}\times{}
  \int_{S^{2}}
  {
    {Y_{\ell_{f}}^{m_{f}}}^{*}\lr{\Omega_{2}}
    {Y_{\lambda}^{\mu}}^{*}\lr{\Omega_{2}}
    Y_{\ell_{i}}^{m_{i}}\lr{\Omega_{2}}
  }
  \dd{\Omega_{2}}
  \\
  &
  {}
  &{}={}&
  \sum_{\lambda = 0}^{\infty}
  \dfrac{4\pi}{2\lambda + 1}
  \sum_{\mu = -\lambda}^{\lambda}
  &&
  \int_{0}^{\infty}
  \int_{0}^{\infty}
  {
    u_{\ell'}\lr{r_{1}; k'}
    \phi_{f}\lr{r_{2}}
    \dfrac{r_{<}^{\lambda}}{r_{>}^{\lambda + 1}}
    \phi_{i}\lr{r_{2}}
    u_{\ell}\lr{r_{1}; k}
  }
  \dd{r_{1}}
  \dd{r_{2}}
  \\
  &
  {}
  &&
  {}
  &&
  {}\times{}
  \int_{S^{2}}
  {
    {Y_{\ell'}^{m'}}^{*}\lr{\Omega_{1}}
    Y_{\lambda}^{\mu}\lr{\Omega_{1}}
    Y_{\ell}^{m}\lr{\Omega_{1}}
  }
  \dd{\Omega_{1}}
  \\
  &
  {}
  &&
  {}
  &&
  {}\times{}
  \int_{S^{2}}
  {
    {Y_{\ell_{f}}^{m_{f}}}^{*}\lr{\Omega_{2}}
    {Y_{\lambda}^{\mu}}^{*}\lr{\Omega_{2}}
    Y_{\ell_{i}}^{m_{i}}\lr{\Omega_{2}}
  }
  \dd{\Omega_{2}}
  .
\end{alignat*}
However, in the s-wave model, we have that that all $\ell, m$ terms (for the
continuum wave) are zero, where therefore all
$Y_{\ell}^{m}\lr{\Omega} = Y_{0}^{0}\lr{\Omega} = \tfrac{1}{\sqrt{4\pi}}$.
It then follows that
\begin{equation*}
  \int_{S^{2}}
  {
    {Y_{\ell'}^{m'}}^{*}\lr{\Omega_{1}}
    Y_{\lambda}^{\mu}\lr{\Omega_{1}}
    Y_{\ell}^{m}\lr{\Omega_{1}}
  }
  \dd{\Omega_{1}}
  =
  \dfrac{1}{\sqrt{4 \pi}}
  \int_{S^{2}}
  {
    {Y_{\ell'}^{m'}}^{*}\lr{\Omega_{1}}
    Y_{\lambda}^{\mu}\lr{\Omega_{1}}
  }
  \dd{\Omega_{1}}
  =
  \dfrac{1}{\sqrt{4 \pi}}
  \delta_{\ell', \lambda}
  \delta_{m', \mu}
\end{equation*}
and as $\ell', m' = 0$, the only non-zero term in the sum is where
$\lambda = \mu = 0$.
Whence, we have that
\begin{alignat*}{3}
  &
  \mel{\vb{k}', \phi_{f}}{\hat{V}_{1, 2}}{\phi_{i}, \vb{k}}
  &{}={}&
  \sqrt{4\pi}
  &&
  \int_{0}^{\infty}
  \int_{0}^{\infty}
  {
    u_{\ell'}\lr{r_{1}; k'}
    \phi_{f}\lr{r_{2}}
    \dfrac{r_{<}^{0}}{r_{>}^{1}}
    \phi_{i}\lr{r_{2}}
    u_{\ell}\lr{r_{1}; k}
  }
  \dd{r_{1}}
  \dd{r_{2}}
  \\
  &
  {}
  &&
  {}
  &&
  {}\times{}
  \int_{S^{2}}
  {
    {Y_{\ell_{f}}^{m_{f}}}^{*}\lr{\Omega_{2}}
    {Y_{0}^{0}}^{*}\lr{\Omega_{2}}
    Y_{\ell_{i}}^{m_{i}}\lr{\Omega_{2}}
  }
  \dd{\Omega_{2}}
  \\
  &
  {}
  &{}={}&
  &&
  \int_{0}^{\infty}
  \int_{0}^{\infty}
  {
    u_{\ell'}\lr{r_{1}; k'}
    \phi_{f}\lr{r_{2}}
    \dfrac{1}{r_{>}}
    \phi_{i}\lr{r_{2}}
    u_{\ell}\lr{r_{1}; k}
  }
  \dd{r_{1}}
  \dd{r_{2}}
  \\
  &
  {}
  &&
  {}
  &&
  {}\times{}
  \int_{S^{2}}
  {
    {Y_{\ell_{f}}^{m_{f}}}^{*}\lr{\Omega_{2}}
    Y_{\ell_{i}}^{m_{i}}\lr{\Omega_{2}}
  }
  \dd{\Omega_{2}}
  \\
  &
  {}
  &{}={}&
  &&
  \int_{0}^{\infty}
  \int_{0}^{\infty}
  {
    u_{\ell'}\lr{r_{1}; k'}
    \phi_{f}\lr{r_{2}}
    \dfrac{1}{r_{>}}
    \phi_{i}\lr{r_{2}}
    u_{\ell}\lr{r_{1}; k}
  }
  \dd{r_{1}}
  \dd{r_{2}}
  \\
  &
  {}
  &&
  {}
  &&
  {}\times{}
  \delta_{\ell_{f}, \ell_{i}}
  \delta_{m_{f}, m_{i}}
\end{alignat*}
and so we may restrict our attention to the case where
$\ell_{f} = \ell_{i}$ and $m_{f} = m_{i}$; that is, where the angular momentum
of the target is left unchanged.
Finally, we have that
\begin{equation*}
  \mel{\vb{k}', \phi_{f}}{\hat{V}_{1, 2}}{\phi_{i}, \vb{k}}
  =
  \mel{k', \phi_{f}}{\hat{V}_{1, 2}}{\phi_{i}, k}
  =
  \int_{0}^{\infty}
  \int_{0}^{\infty}
  {
    u_{\ell'}\lr{r_{1}; k'}
    \phi_{f}\lr{r_{2}}
    \dfrac{1}{r_{>}}
    \phi_{i}\lr{r_{2}}
    u_{\ell}\lr{r_{1}; k}
  }
  \dd{r_{1}}
  \dd{r_{2}}
\end{equation*}
yielding the result as required.

\section{Reduced CCC Code}
\label{sec:reduced-ccc-code}

\end{document}
